% Linux Kniha kouzel, šablona formátu PDF
% Copyright (c) 2019 Singularis <singularis@volny.cz>
%
% Toto dílo je dílem svobodné kultury; můžete ho šířit a modifikovat pod
% podmínkami licence Creative Commons Attribution-ShareAlike 4.0 International
% vydané neziskovou organizací Creative Commons. Text licence je přiložený
% k tomuto projektu nebo ho můžete najít na webové adrese:
%
% https://creativecommons.org/licenses/by-sa/4.0/
%
% This program is free software; you can redistribute it and/or modify
% it under the terms of the GNU General Public License as published by
% the Free Software Foundation; either version 2 of the License, or
% (at your option) any later version.
%
% This program is distributed in the hope that it will be useful,
% but WITHOUT ANY WARRANTY; without even the implied warranty of
% MERCHANTABILITY or FITNESS FOR A PARTICULAR PURPOSE. See the
% GNU General Public License for more details.
%
% You should have received a copy of the GNU General Public License
% along with this program; if not, write to the Free Software
% Foundation, Inc., 51 Franklin St, Fifth Floor, Boston, MA  02110-1301  USA
%
{{ZAČÁTEK KNIHY}}
\documentclass[10pt,final]{book}
% 1. skupina balíčků:
\usepackage[utf8]{inputenc}
\usepackage[czech]{babel}
%
{{POKUD JE FORMÁT pdf-a4}}
\usepackage[pdftex,a4paper,inner=2.5cm,outer=2cm,top=1.5cm,bottom=2cm]{geometry} % + showframe pro ladění
\newcommand{\inicializovatnastaveni}{%
    \newlength{\sirkakodu}\setlength{\sirkakodu}{0.9\textwidth} % šířka kódu v příkladech
    \setlength{\columnsep}{16pt}%
    \setlength{\columnseprule}{1pt}% pro balík {multicols}
    \renewcommand*{\columnseprulecolor}{\color{sedamezisloupci}}%
    \newenvironment{odstavce}{\begin{adjustwidth}{}{}\nastavitradkovaniodstavcu{}}{\end{adjustwidth}}
    \newenvironment{obsahdosloupcu}{\begin{multicols*}{2}}{\end{multicols*}}
    \newcommand*{\zapnoutrezimlicence}{\begin{multicols*}{2}\tiny\renewenvironment*{odstavce}{\begin{adjustwidth}{}{}}{\end{adjustwidth}}}
    \newcommand*{\vypnoutrezimlicence}{\end{multicols*}}
    \newcommand*{\nadpisobsahu}{\mbox{PŘEHLED (OBSAH)}}%
    \newcommand*{\nastavitradkovaniodstavcu}{\renewcommand*{\baselinestretch}{1.2}\selectfont}
}%
{{KONEC POKUD}}
%
{{POKUD JE FORMÁT pdf-b5}}
\usepackage[pdftex,b5paper,inner=2cm,outer=1.5cm,top=1.5cm,bottom=2cm]{geometry} % + showframe pro ladění
\newcommand{\inicializovatnastaveni}{%
    \newlength{\sirkakodu}\setlength{\sirkakodu}{0.95\textwidth}
    \newenvironment{odstavce}{\begin{adjustwidth}{}{}}{\end{adjustwidth}}
    \newenvironment{obsahdosloupcu}{}{}
    \newcommand*{\zapnoutrezimlicence}{\begin{multicols*}{2}\tiny\renewenvironment*{odstavce}{\begin{adjustwidth}{}{}}{\end{adjustwidth}}}
    \newcommand*{\vypnoutrezimlicence}{\end{multicols*}}
    \newcommand*{\nadpisobsahu}{\mbox{PŘEHLED} \mbox{(OBSAH)}}%
    \newcommand*{\nastavitradkovaniodstavcu}{}%
}%
{{KONEC POKUD}}
%
{{POKUD JE FORMÁT pdf-a5}}
\usepackage[pdftex,a5paper,inner=1.5cm,outer=1cm,top=1.5cm,bottom=1.5cm]{geometry} % + showframe pro ladění
\newcommand{\inicializovatnastaveni}{%
    \newlength{\sirkakodu}\setlength{\sirkakodu}{0.98\textwidth}
    \newenvironment{odstavce}{\begin{adjustwidth}{}{}}{\end{adjustwidth}}
    \newenvironment{obsahdosloupcu}{}{}
    \newcommand*{\zapnoutrezimlicence}{\begin{blok}\tiny\renewenvironment*{odstavce}{\begin{adjustwidth}{}{}}{\end{adjustwidth}}}
    \newcommand*{\vypnoutrezimlicence}{\end{blok}}
    \newcommand*{\nadpisobsahu}{\mbox{PŘEHLED} \mbox{(OBSAH)}}%
    \newcommand*{\nastavitradkovaniodstavcu}{}%
}%
{{KONEC POKUD}}
%
%\usepackage[B1,LY1,T1]{fontenc}
%\usepackage[T1]{fontenc}
\usepackage[IL2,T1]{fontenc}

% 2. skupina balíčků:
\usepackage{mathabx}        % symboly (zejména \dlsh)
\usepackage{textcomp}       % musí být načtený přednostně; vyžadovaný balíčkem {gensymb}
\usepackage[pdftex]{xcolor} % barevné písmo

% 3. skupina balíčků:
\let\degree=\undefined      % nouzové řešení konfliktu (použít \degree z {gensymb}, ne z {mathabx})
\usepackage{changepage}     % prostředí {adjustwidth}
\usepackage{dashbox}        % příkazy \dbox{} a \dashbox{}
\usepackage{fancyhdr}       % formátování záhlaví a zápatí
\usepackage[bottom]{footmisc}% aby se poznámky pod čarou správně umístily
\usepackage{gensymb}        % symboly jako např. \degree
\usepackage[pdftex]{graphicx}% vkládání obrázků
\usepackage{ifthen}         % příkaz \ifthenelse a související konstrukce
\usepackage{multicol}       % prostředí {multicols} a {multicols*}
\usepackage{tabu}           % lepší tabulky (prostředí {tabu})
\usepackage{tcolorbox}      % boxy s pozadím, kulatými okraji a podobně
\usepackage{titlesec}       % formátování nadpisů kapitol
\usepackage[titles]{tocloft}% formátování přehledu („obsahu“)
\usepackage{verbatim}       % zejm. prostředí {comment}, ale také {verbatim}

%
% NASTAVENÍ BALÍČKŮ
% ============================================================================
% {dashbox}:
\setlength{\fboxrule}{0.25pt}   % tloušťka čáry
\setlength{\dashdash}{1pt}      % délka čárky
\setlength{\dashlength}{3pt}    % vzdálenost od začátku čárky k začátku další

% {fancyhdr}:
\fancypagestyle{normalni}{%
    \fancyhead{}\fancyfoot{}%
    \fancyfoot[LE,RO]{\thepage}%
    \renewcommand*{\headrulewidth}{0pt}%
    \renewcommand*{\footrulewidth}{0pt}%
}

% {titlesec}:
\titleclass{\chapter}{top}\relax%
\newcommand{\chapterbreak}{\clearpage}% začínat kapitolu na nové straně
\titleformat{\chapter}[hang]{\normalfont\LARGE\bfseries}{\settowidthwithminimum{\sirkastitkutmp}{\mbox{\thechapter}}{20pt}\makebox[\sirkastitkutmp][l]{\thechapter}}{0.33em}{}\relax
\titlespacing*{\chapter}{0pt}{-1ex}{2ex plus2ex minus0.5ex}\relax

\titleformat{\section}[hang]{\normalfont\Large\bfseries}{\settowidthwithminimum{\sirkastitkutmp}{\mbox{\thesection}}{20pt}\makebox[\sirkastitkutmp][l]{\thesection}}{0.33em}{}\relax
\titlespacing*{\section}{0pt}{2ex plus0.5ex minus0ex}{1ex}\relax

\titleformat{\subsection}[hang]{\normalfont\large\bfseries}{}{0pt}{%
%    \filcenter%
    \begin{blok}%
        \setlength{\unitlength}{1em}%
        \begin{picture}(1.7,1)(0,0.16)\relax%
            \put(0,-0.2){{\color{svetleseda}\rule{\textwidth}{1.4\unitlength}}}%
            \put(0.2,0){\makebox(1,1){\arabic{subsection}}}%
            \put(0.7,0.5){\circle{1.2}}%
        \end{picture}%
    \end{blok}%
%    \filcenter\arabic{subsection}~%
}\relax
\titlespacing*{\subsection}{0pt}{4ex plus1ex minus0ex}{1ex}\relax

\assignpagestyle{\chapter}{normalni}

% {tocloft}:
\setlength{\cftpartindent}{0pt}
\setlength{\cftchapindent}{0pt}
\setlength{\cftsecindent}{1em}
\setlength{\cftsubsecindent}{2em}
%\setlength{\cftsubsubsecindent}{3em}

% {xcolor}:
\definecolor{cervena}{rgb}{1,0,0}
\definecolor{svetleseda}{gray}{0.9}
\definecolor{seda}{gray}{0.5}
\definecolor{sedamezisloupci}{gray}{0.75}

% Inicializace nastavení závislých na formátu stránky:
\inicializovatnastaveni

%
% DEFINICE DÉLKOVÝCH REGISTRŮ
% ============================================================================
\newlength{\sirkacislazaklinadla}\setlength{\sirkacislazaklinadla}{23pt}
\newlength{\sirkazbytku}
\newlength{\odsazeni}\setlength{\odsazeni}{1em}
\newlength{\minsirkastitku}\setlength{\minsirkastitku}{2cm}
\newlength{\sirkastitkutmp}


%
% KOREKCE DĚLENÍ SLOV
% ============================================================================
\hyphenation{
    Git-Hub Git-Hubu Git-Hube Git-Hubem
    ffmpeg ffmpegu
    make
% úprava dělení slov kvůli sazbě:
    trou-fne
}
%
% Další definice
% ============================================================================
\newenvironment*{blok}{}{}

\newcommand*{\kapitola}[2][]{\chapter{#2}}
\newcommand*{\sekce}[2][]{\section{#2}}
\newcommand*{\podsekce}[2][]{\subsection{#2}}

\newcommand*{\budepoprve}{%
    \renewcommand{\nepoprve}[1]{%
        \renewcommand{\nepoprve}[1]{####1}%
    }%
}
\newcommand*{\dopln}[1]{{\fontfamily{qcr}\fontseries{m}\fontshape{it}\selectfont\underline{{#1}}}}%
\newcommand*{\moznyzlom}{{\fontseries{m}\fontshape{n}\selectfont\textcolor{seda}{\-}}}%
\newcommand{\nepoprve}[1]{#1}
\newcommand*{\tritecky}{{\fontfamily{cmss}\fontseries{m}\fontshape{n}\selectfont\normalsize\mbox{...}}}%
% {} \big \Big \bigg \Bigg
\newcommand*{\volitelnyzacatek}{\textcolor{seda}{\ensuremath{\big[}}}%
\newcommand*{\volitelnykonec}{\textcolor{seda}{\ensuremath{\big]}}}%
\newcommand*{\carriagereturn}{\mbox{}\dlsh}%
\newenvironment*{odsazenyodstavec}[1]{\vspace{1ex}\begin{adjustwidth}{#1\odsazeni}{}}{\end{adjustwidth}\vspace{1ex}}%
% \settowidthwithminimum{\registr}{text}{minimum}
\newcommand*{\settowidthwithminimum}[3]{\settowidth{#1}{#2}\ifthenelse{\lengthtest{#1<#3}}{\setlength{#1}{#3}}{}}
\newcommand*{\stitek}[1]{{%
    \settowidth{\sirkastitkutmp}{#1}%
    \ifthenelse{\lengthtest{\sirkastitkutmp<\minsirkastitku}}{\setlength{\sirkastitkutmp}{\minsirkastitku}}{}%
    \mbox{\tcbox[boxrule=0.5pt,size=fbox,arc=1mm]{\makebox[\sirkastitkutmp]{\rule[-0.45ex]{0pt}{1em}#1}}}%
}}

%
% \zaklinadlo[#1=(rezervováno)]{#2=číslo zaklínadla}{#3=titulek (volitelný) + \footnotemark}{#4=\footnotetext...}{#5=řádky a akce zaklínadla}
% ====================================
\newcommand{\zaklinadlo}[5][]{{%
    % Definovat \radekzaklinadla{text}:
    \newcommand*{\radekzaklinadla}[1]{%
        % Ukončit předchozí řádek:
        \nepoprve{{%
            \fontencoding{IL2}\fontfamily{bsk}\fontseries{m}\fontshape{n}\selectfont%
            \textcolor{seda}{$\carriagereturn$}\\\relax%
        }}%
        % Vypsat samotný řádek zaklínadla (nutný „strut“ kvůli výšce řádku a orámování):
        {\rule{0pt}{1.5ex}##1}%
    }%
    % Definovat \akcezaklinadla{text}
    \newcommand*{\akcezaklinadla}[1]{\radekzaklinadla{\ensuremath{\Rightarrow}~\rmfamily{}##1}}%
    \setlength{\sirkazbytku}{\textwidth}%
    \addtolength{\sirkazbytku}{-\sirkacislazaklinadla}%
    \noindent%
    \parbox{\textwidth}{\raggedright%
        \ifthenelse{\equal{#2}{0}}{}{%
            % Vygenerovat vodicí čáru nad zaklínadlem:
            \ifthenelse{\equal{#2}{1}}{}{%
                \makebox[0pt][l]{{\color{seda}\rule[2ex]{5cm}{0.25pt}}}%
            }%
            % Vygenerovat číslo zaklínadla:
            \makebox[\sirkacislazaklinadla][l]{\##2}%
            % Titulek zaklínadla:
            \parbox[t]{\sirkazbytku}{\raggedright{#3}}%
            \par\vspace{1ex}%
        }\raggedleft%
        \color{seda}%
        \dashbox{\color{black}\parbox{\sirkakodu}{\raggedright%
            \fontfamily{qcr}\fontseries{m}\fontshape{n}\selectfont\budepoprve%
            #5%
        }}%
    }#4\par\vspace{2ex plus0.5ex minus0.25ex}%
}}
% qcr = TeX Gyre Cursor

\renewcommand*{\rmdefault}{cmr}
\renewcommand*{\sfdefault}{cmss}
\renewcommand*{\ttdefault}{qcr}

\begin{document}%
%
% TITULNÍ STRANA
%
\pagestyle{empty}%
\vspace*{0.05\textheight}%
\begin{center}\fontfamily{cmss}\selectfont\bfseries\Huge\scalebox{1.5}{LINUX}\\[0.01\textheight]\Large Kniha kouzel\end{center}%
\vspace{-2cm}\vfill%
\begin{blok}%
    \setlength{\unitlength}{0.5\textwidth}%
    \begin{picture}(2,1.4)%
        \put(0,0){\includegraphics[width=0.5\textwidth]{../pdf-spolecne/_obrazky/logo-knihy-velke.png}}%
        \put(1,1.4){\parbox[t][0.7\textwidth][c]{0.5\textwidth}{\centering\itshape%
            Praktická sbírka krátkých řešených příkladů,\\%
            také známá jako „konverzační slovník linuxštiny“\\[0.02\textwidth]%
            Veškerá moc příkazové řádky/příkazového řádku přehledně,
            pro~začátečníky i pokročilé}}%
    \end{picture}%
\end{blok}%
\par\vfill%
\begin{center}\mbox{https://github.com/singularis-mzf/linux-spellbook}\\%
verze: {{JMÉNO VERZE}}%
\\[0.03\textheight]\textcopyright\,2019 Singularis a ostatní přispěvatelé\end{center}%
\clearpage%
%
% DRUHÁ STRANA
%
\pagestyle{empty}%
{\noindent\Huge Linux --- Kniha kouzel\par}%
\vspace{2ex}\noindent%
\textcopyright~2019 Singularis a ostatní přispěvatelé%

\vspace{2ex}\noindent%
Toto dílo \quotedblbase Linux: Kniha kouzel\textquotedblleft{} podléhá licenci
Creative Commons Attribution-ShareAlike 4.0 International. Úplný text licence
je přiložen. Pro jeho zobrazení na počítači navštivte:
\begin{center}\ttfamily%
http://creativecommons.org/licenses/by-sa/4.0/
\end{center}

\noindent\textbf{Použitá díla jiných autorů:}
\begin{itemize}%
\item obrazky/ve-vystavbe.png \textcopyright~2010, Sarang (volné dílo podle německých zákonů)
\item tučňák Tux \textcopyright~Larry Ewing; \textcopyright~Garrett LaSage (licence CC0, upraveno)
\end{itemize}%
\vfill%
\clearpage%
%
% Nastavit styl záhlaví a zápatí
\pagestyle{normalni}%
% Vysázet obsah
\begin{obsahdosloupcu}%
\renewcommand*{\contentsname}{\nadpisobsahu}%
\tableofcontents%
\end{obsahdosloupcu}%
%
%
\clearpage\mbox{}\par\vfill%
\begin{center}{\normalfont\LARGE\bfseries\makebox[35pt][l]{}Typografické konvence\par}\vspace{2ex}%
%\titleformat{\chapter}[hang]{\normalfont\LARGE\bfseries}{\makebox[35pt][l]{\thechapter}}{5pt}{}\relax
%\titlespacing*{\chapter}{0pt}{-1ex}{2ex plus2ex minus0.5ex}\relax
%\chapter*{\begin{center}Typografické konvence\end{center}}%
\setlength{\abovetabulinesep}{1ex}\setlength{\belowtabulinesep}{\abovetabulinesep}%
\begin{tabu}{|X[-1,l]|X[2,l]|}%
{\ttfamily sudo apt-get update}&%
    Texty tištěné normálním neproporcionálním písmem máte přepsat
    přesně tak, jak jsou napsány.\\%
{\ttfamily\itshape\underline{jméno-souboru}}&%
    Za skloněné podtržené části zaklínadla máte doplnit konkrétní hodnoty.\\%
{\textcolor{seda}{\ensuremath{\dlsh}}}&%
    Dělí-li se zaklínadlo na víc řádků či variant,
    je každá z nich ukončena šedým symbolem odřádkování; není-li na konci řádku tento symbol,
    jde jen o zalomení řádku v knize.\\%
{\ttfamily ABC=\,\textcolor{seda}{{\fontfamily{qcr}\fontseries{m}\fontshape{n}\selectfont\char`\\}}}&%
    Ve výjimečných případech, kdy bylo nutno řádek zalomit v místě,
    kam napatří žádné bílé znaky, je tato skutečnost vyznačena nepatrným šedým zpětným lomítkem.
    Takové zalomení zcela ignorujte, nevkládejte ani mezeru.\\%
\textcolor{seda}{$\big[$}{\ttfamily{-}{-}volitelny []}\textcolor{seda}{$\big]$}&%
    Nepovinné a doplňkové části zaklínadla jsou uzavřeny do vysokých šedých
    hranatých závorek, jasně odlišitelných od obyčejných hranatých závorek.\\%
{\ttfamily\itshape\underline{další-soubor}}\textcolor{seda}{...}&%
    Části zaklínadla, které je možno uvést vícekrát, jsou vyznačeny šedou trojtečkou.\\%
{\ttfamily{}@\textvisiblespace\textquotedbl}&%
    Ke zdůraznění místa, kam je nutno zapsat právě jednu obyčejnou mezeru, se v zaklínadlech
    používá symbol {\ttfamily\textvisiblespace}.\\%
$\Rightarrow$ Klikněte.&%
    Ve výjimečných případech, kdy je místo napsání něčeho potřeba něco vykonat,
    je tato instrukce vyjádřena šipkou a textem v obyčejném písmu.\\%
\textcolor{seda}{\guillemotright}&%
    Tímto šedým symbolem se vyznačuje, že na dané místo musíte zapsat tabulátor
    (mezery nestačí).\\%
{\ttfamily\textcolor{seda}?}&%
    Není-li řešení dané úlohy autorovi při psaní známo, uvede místo řádku zaklínadla
    pouhý otazník. To je podnět k tomu, aby bylo řešení nalezeno a doplněno.
    Hojně se vyskytuje v rozpracovaných kapitolách, ale u náročných úloh na něj
    můžete příležitostně narazit i ve vydaných verzích.\\%
\end{tabu}%
\end{center}\par\vfill\mbox{}%
%
% Nastavit znak pro „dělení slov“ v kódu na zpětné lomítko
{%
    \fontfamily{qcr}%
    \fontseries{m}\selectfont\hyphenchar\font=`\\%
    \fontseries{b}\selectfont\hyphenchar\font=`\\%
    \fontseries{bx}\selectfont\hyphenchar\font=`\\%
    \fontshape{it}\fontseries{m}\selectfont\hyphenchar\font=`\\%
    \fontseries{b}\selectfont\hyphenchar\font=`\\%
    \fontseries{bx}\selectfont\hyphenchar\font=`\\%
}%
%
\clearpage%
\raggedbottom%
{{ZAČÁTEK KAPITOLY}}
{{TĚLO KAPITOLY}}
{{KONEC KAPITOLY}}
\end{document}
{{KONEC KNIHY}}
